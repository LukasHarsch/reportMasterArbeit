%=====================================================================
\chapter{Hilfsprogramme}
\label{chap:hilfsprogramme}
%=====================================================================
Das vorliegende Kapitel stellt einige Hilfsprogramme vor, die f�r die Arbeitserstellung hilfreich sein k�nnen. Die Aufstellung ist keineswegs komplett und erhebt nicht den Anspruch auf Vollst�ndigkeit. Vielmehr sollen hier Anregungen gegeben werden, welche Hilfsmittel hilfreich sein k�nnen.
\section{Graphen erstellen}
\label{sec:graphenerstellen}
Graphen oder Diagramme k�nnen mit einer Vielzahl von Programmen erstellt werden. Hierbei haben alle Programme mittlerweile vielf�ltige M�glichkeiten, professionell Diagramme zu gestalten. Die Programme unterscheiden sich aber in der Bedienung und haben verschiedenste Vor- und Nachteile. Am Institut hat sich das frei verf�gbare Programm ``gnuplot''~\cite{gnuplot} durchgesetzt und ist unter Linux verf�gbar. Unter dem aufgef�hrten Link findet man unz�hlige Beispiele. Diagramme k�nnen direkt als ``eps'' gespeichert werden, welche optimal f�r die Einbindung in das Latexdokument geeignet sind.
\section{Latex-Editoren}
\label{sec:latexeditoren}
Unter Linux gibt es eine Vielzahl von Texteditoren, die frei verf�gbar sind. Genannt werden sollten auf jeden Fall ``vi'' und ``kile'', jedoch k�nnen alle uneingeschr�nkt empfohlen werden. Hiermit kann eine ``tex``-Datei editiert werden. Wer die Befehle f�r z.B. Latex nicht auswendig kennt, kann auch ein Hilfsprogramm wie ''kile`` verwenden~\cite{kile}, welches unter Linux verf�gbar ist.
\section{Latexhilfen}
\label{sec:latexhilfen}
Die offizielle Latexseite~\cite{latex} kann bei der Erstellung einer Arbeit in Latex helfen. Zudem gibt es eine Vielzahl von Foren und weiteren Seiten, die zu speziellen Fragen in Latex weiterhelfen. Hierbei werden bewusst keine weiteren Seiten aufgez�hlt. Eine kurze Internetrecherche hilft bei einem speziellen Problem immer weiter.
\section{Bilder erstellen}
\label{sec:bildererstellen}
Unter Linux kann ''gimp``~\cite{gimp} f�r die Bildbearbeitung empfohlen werden. Eine Vielzahl von Hilfen finden sich im Netz.
\section{Weitere Programme}
\label{sec:weitereprogramme}
F�r viele Arbeiten werden weitere Programme ben�tigt, deshalb k�nnen unm�glich alle Varianten aufgez�hlt werden.
Deshalb werden hier nur kurz weitere Hilfswerkzeuge vorgestellt:
\begin{enumerate}
 \item perl - Eine m�chtige Programmiersprache, die zur Laufzeit kompiliert
 \item convert - Ein Konvertierungsprogramm f�r Bilder
 \item rsync - Tool zum Aktualisieren von Dateien und f�r den Dateitransfer
 \item ssh - Netzwerkprotokoll
 \item tail, head, watch - N�tzliche Shellprogramme
 \item gcc - die GNU Kompilierungskollektion
 \item proE - CAD Programm
 \item OpenOffice - Programm zum Schreiben von Texten, Tabellenkalkulation etc.
 \item gwenview - Bildbetrachtungstool
 \item kcalc - Taschenrechner f�r Linux
 \item evince, gv, okular - Betrachten von pdf's
 \item okular - Betrachten von dvi's
 \item dvi2ps/pdf - Konvertierungsprogramm siehe Namen
 \item tkdiff - Programm f�r den visuellen Dateivergleich
 \item hg - Freies, m�chtiges Quellcode Management Programm
\end{enumerate}
Die Liste wird gerne fortgesetzt. Bitte einfach den Tipp an einen IHS-Mitarbeiter geben, oder am besten gleich eine kleine Beschreibung schreiben f�r eine gesondertes Unterkapitel.

