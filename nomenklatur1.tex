%=====================================================================
\chapter*{Nomenklatur\markboth{Nomenklatur}{Nomenklatur}}
\addcontentsline{toc}{chapter}{Nomenklatur}
%=====================================================================

%%%
%%%

\setlength\LTleft{0pt}
%
\subsection*{Lateinische Buchstaben}
\begin{longtable}{@{\extracolsep{\LTleft}}p{20.0mm}p{25.0mm}l}
% $A$ & $\mathrm{m^2}$ & Fl\"{a}che\\
% $a$ & $\%$ & Abstand\\
% $a$ & $\mathrm{m/s^2}$ & Beschleunigung\\
% $B$ & $\mathrm{N/m^2}$ & Druckamplitude\\
% $c$ & $\mathrm{m/s}$ & Geschwindigkeit\\
% $D$ & $\mathrm{m}$ & Durchmesser\\
% $F$ & $\mathrm{N}$ & Kraft\\
% $f$ & $\mathrm{s^{-1}}$ & Frequenz\\
$g$ & $\mathrm{m/s^2}$ & Erdbeschleunigung\\
$H$ & $\mathrm{m}$ & (Fall)H\"{o}he\\
%$h$ & $\mathrm{\%}$ & relative Kanalh\"{o}he\\
%$k$ & $\mathrm{W/(mK)}$ & thermische Konduktivit\"{a}t\\

\end{longtable}

\subsection*{Griechische Buchstaben}
\begin{longtable}{@{\extracolsep{\LTleft}}p{20.0mm}p{25.0mm}l}
%$\alpha$ & $\�$ & absoluter Str\"{o}mungswinkel\\
%$\beta$ & $\�$ & relativer Str\"{o}mungswinkel\\
$\Delta$ & $\%$ & Abweichung\\
%$\Delta\gamma$ & $\�$ & Schlie{\ss}winkel\\
%$\delta_{ij}$ & - & Kronecker-Delta\\
$\epsilon$ & $\mathrm{m^2/s^3}$ & isotrope Dissipationsrate\\
% $\eta$ & $\%$ & Wirkungsgrad\\ %neu
% $\eta$ & $\mathrm{kg/(ms)}$ & dynamische Viskosit"at\\
% $\theta$ & $\�$ & Umfangswinkel\\
% $\vartheta$ & $\�\mathrm{C}$ & Temperatur \\
%$\kappa$ & $\mathrm{-}$ & Isentropenexponent\\

\end{longtable}

\subsection*{Tiefgestellte Indizes}
\begin{longtable}{@{\extracolsep{\LTleft}}p{20.0mm}l}
$1,2$ & Eintritt, Austritt\\
% $\mathrm{a}$ & Amplitude\\ %neu
% $\mathrm{ab}$ & abgegeben\\ %neu
$\mathrm{abs}$ & absolut\\ %neu
%$\mathrm{ax}$ & in axialer Richtung\\
%$\mathrm{Austritt}$ & Austrittsrand des Rechengebiets\\
%$\mathrm{choke}$ & Sperrgrenze\\
%$\mathrm{design}$ & Auslegungspunkt\\
%$\mathrm{Eintritt}$ & Eintrittsrand des Rechengebiets\\
% $\mathrm{eff}$ & effektiv\\ %neu
% $\mathrm{eul}$ & Euler\\ %neu
%$\mathrm{hinten}$ & hintere Spaltkavit\"{a}t\\
$\mathrm{hyd}$ & hydraulisch\\ %neu
%$\mathrm{IN}$ & Eintritt\\
% $\mathrm{i,j}$ & Z\"{a}hlindex\\ %neu
% $\mathrm{ideal}$ & ideal\\ %neu
%$\infty$ & ungest"orte Str"omung am Rand\\
%$\mathrm{n}$ & normal\\
%$\mathrm{t}$ & tangential
\end{longtable}

% \subsection*{Hochgestellte Indizes}
% \begin{longtable}{@{\extracolsep{\LTleft}}p{20.0mm}l}
% $*$ & skaliert (entsprechend Pitchverh\"{a}ltnis)
% \end{longtable}

% \subsection*{Dimensionslose Kennzahlen}
% \begin{math}
% \begin{array}{@{\extracolsep{\LTleft}}lcll}
% y^{+} & := & y\dfrac{v_{\tau}}{\nu} & \mbox{dimensionsloser
% Wandabstand}
% %\Ma & := & \|\vec{V}\|/a & \mbox{Mach-Zahl}\\
% %\Re & := & \|\vec{V}\|L/\nu & \mbox{Reynolds-Zahl}
% \end{array}
% \end{math}

% \subsection*{Dimensionslose Kennzahlen}
% \begin{longtable}{@{\extracolsep{\LTleft}}p{20.0mm}l}
% $y^{+}  :=  y\dfrac{v_{\tau}}{\nu}$ & dimensionsloser Wandabstand
% \end{longtable}


\subsection*{Abk�rzungen}
\begin{longtable}{@{\extracolsep{\LTleft}}p{20.0mm}l}
BP & Betriebspunkt\\ %neu
% BPF & Blade Passing Frequency\\ %neu
% $\mathrm{DNS}$ & Direkte Numerische Simulation\\
% $\mathrm{FEM}$ & Finite-Elemente-Methode  \\
% $\mathrm{FVM}$ & Finite-Volumen-Methode \\
% $\mathrm{GPF}$ & Gate Passing Frequency\\ %neu
% $\mathrm{LES}$ & Large Eddy Simulation\\
% $\mathrm{RANS}$ & Reynolds Averaged Navier Stokes\\ %neu
% $\mathrm{RSI}$ & Rotor-Stator-Interaktion \\
DT & (Teil)Rechengebiet Saugrohr \\


\end{longtable}


%\subsection*{Operatoren, Funktionen und Symbole}
%\begin{longtable}{@{\extracolsep{\LTleft}}p{20.0mm}l}
%$|\bullet|$ & Betrag eines Skalars\\
%$\|\bullet\|$ & Norm (L"ange) eines Vektors\\
%d & totales bzw.\ vollst"andiges Differential\\
%$\deter(\bullet)$ & Determinante\\
%$\diverg(\bullet)$ & Divergenz einer tensoriellen Gr"o"se\\
%D & materielles (substantielles) Differential\\
%$\grad(\bullet)$ & Gradient einer tensoriellen Gr"o"se\\
%$\delta$ & unvollst"andiges Differential\\
%$\partial$ & partielles Differential\\
%$\partial V$ & st"uckweise glatter Rand eines Volumens\\
%$\nabla$ & Nablaoperator\\
%$\cdot$ & Skalarprodukt (inneres Produkt)\\
%$\times$ & Kreuzprodukt ("au"seres Produkt)\\
%\end{longtable}
