%=====================================================================
\chapter{Allgemeine Hilfen}
\label{chap:allgemeinehilfen}
%=====================================================================
Das vorliegende Kapitel soll dem Studenten eine kleine Anleitung geben, wie man eine wissenschaftliche Arbeit schreibt.\\
Daf�r werden einige Tipps gegeben und auf m�gliche Fehlerquellen hingewiesen, die man unbedingt vermeiden sollte. Bitte dieses Kapitel aufmerksam durchlesen und in der eigenen Arbeit umsetzen.
\section{Gliederung der Arbeit}
Zu Beginn muss die Arbeit sinnvoll gegliedert werden. Die Arbeiten enthalten immer eine Einleitung und eine Zusammenfassung mit anschlie�endem Ausblick. Je nach Art der Arbeit beschreibt der Mittelteil die eigentlich vorgenommenen Untersuchungen. Eine beispielhafte Gliederung kann folgenderma�en aussehen:
\begin{enumerate}
 \item Einleitung
 \item Theoretische Herleitung
 \item Modellierung und Simulationsaufbau
 \item Ergebnisse
 \item Zusammenfassung
 \item Ausblick
\end{enumerate}
\section{Hilfestellung f�r die Gliederung der Arbeit}
Im allgemeinen k�nnen folgende Fragen helfen, wissenschaftliche Texte zu schreiben:
\begin{enumerate}
 \item Um welchen Themenbereich handelt es sich?
 \item Warum ist es wichtig, sich mit diesem Themenbereich zu befassen?
 \item Was hat die bisherige Forschung zu diesem Themenbereich an Erkenntnissen gewonnen?
 \item Wie wurden die vorhandenen Erkenntnisse gewonnen?
 \item Welche Fragen sind bislang offen geblieben?
 \item Welche dieser offenen Fragen ist Gegenstand der vorliegenden Untersuchung?
 \item Wie wurden die neuen Erkenntnisse gewonnen?
 \item Welche neuen Erkenntnisse wurden gewonnen? 
 \item Wie sind die neuen Erkenntnisse im Zusammenhang mit den bereits vorhandenen Erkenntnissen einzusch�tzen?
\end{enumerate}
Die Fragen sind sehr ausf�hrlich gestellt. Bachelorarbeiten k�nnen und m�ssen nicht alle Fragen beantworten. Masterarbeiten k�nnen schon neue Erkenntnisse gewinnen. Die Fragen sollen bei der Gliederung der Arbeit helfen und k�nnen die Inhalte von Einleitung und Theorieteil meist gut vorgeben.\\
\section{Kleine Schreibregeln}
Generell sollten die Kapitel keine Vorw�rtsreferenzen enthalten, die explizit den Leser zwingen, zwischendurch ein Kapitel weiter zu springen.\\
�blicherweise stellt man die verwendeten Theorien und Gleichungen dar, die mit dem Thema zu tun haben. Dinge, die man nicht ben�tigt, geh�ren nicht in die Arbeit. Das w�re unzul�ssiges F�llen von Seiten, und macht das Lesen der Arbeit schwer.\\
Die oben beschriebene Kapitelstruktur schlie�t einen chronologischen Berichtstil aus. Dieser wird in wissenschaftlichen Texten auch nicht gepflegt.\\
Die folgende �bersicht gibt einige Regeln wieder, die beachtet werden sollten:
\begin{enumerate}
 \item Der Text wird immer in der Gegenwart geschrieben.
 \item Keine Vorw�rtsreferenzen, bei denen der Leser springen muss, um den Text zu verstehen.
 \item Nach jeder �berschrift kommt Text.
 \item Unbestimmte Pronomen wie ``man'' sollte nach M�glichkeit vermieden werden.
 \item Unterkapitel machen nur dann Sinn, wenn diese mindestens zu zweit vorkommen.
 \item Aktivkonstruktionen machen den Text zumeist besser leserlich.
 \item Lange Schachtels�tze sind schwierig zu lesen und verkomplizieren den zumeist ohnehin schwierigen und schwer verst�ndlichen Sachverhalt unn�tigerweise �ber das �bliche Ma� hinaus und k�nnen zudem unter Umst�nden zu Aussagen f�hren, die im besten Falle nur zweideutig sind.
 \item Umgangssprache ist tabu. Auch das Verwenden von beschreibenden Adjektiven sollte vermieden werden, wie z.B. ``das Ergebnis sieht sch�n aus''.
 \item F�llw�rter sind unn�tig. Beispiel: ``quasi, eigentlich, etwa, sollte'' etc.
 \item Anglizismen sind kein guter Stil
 \item Wissenschaftlich hei�t nicht unverst�ndlich.
\end{enumerate}