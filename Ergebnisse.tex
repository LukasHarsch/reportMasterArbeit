%=====================================================================
\chapter{Ergebnisse}
\label{sec:Ergebnisse}
%=====================================================================
TODO

\section{Baseline}
TODO
%
%
\section{Autoencoder}
Dieses Kaptil stellt zuerst einmal das Verhalten eines VAE beispielhaft an dem MNIST Datensatz dar. MNIST ist ein Datensatz aus Bildern von handgeschriebenen Zahlen zwischen Null und Neun. Jedes Bild hat dabei 28x28 Pixel. Der Datensatz beinhaltet 60000 Bildern f�r das Training und 10000 f�r die Evaluation. Anschlie�end werden die Ergebnisse der Dataaugmentation f�r den Datensatz dr Spektorgramme pr�sentiert. Dar�ber hinaus wird gezeigt, welchen Einfluss die Dataaugmentation auf das Klassifikationsergebnis hat.\\
\\
%
Tabelle \ref{VAEencArchitecturMNIST} und \ref{VAEdecArchitecturMNIST} zeigen die Encoder und Decoder Architektur des VAE f�r den MNIST Datensatz. Der Encoder besitzt zu Begin vier Convolutional Ebenen mit steigender Filteranzahl von 32, 64 und 128. Zus�tlich wird in Striding von $2x2$, wodurch der Input gedownsampled wird um den Rechenaufwand zu verringern. In der vierten Conv Ebene wird kein Striding mehr angewandt. Als Aktivierungsfunktion wird in allen Conv Ebenen die ReLU Funktion verwendet. Auf die Conv Ebenen folgt jeweils eine Dropout Ebene um die Generalisierung zu erh�hen. Das Resultat der Conv Ebenen sind $4\times4\times128$ Feature Maps. Diese werden in der Flatten Ebene zu einem Vektor umgeformt und durch eine vollverkettete Ebene mit 512 Units geschickt. Zuletzt muss der Latentraum $z$ erzeugt werden. Dieses wird wie in Kapitel X beschreiben als Gausverteilung definiert. Eine Gausverteilung wird �ber zwei Parameter den Erwartungswert $\mu(X)$ und die Varianz $\sigma^2(X)$ beschreiben. Dazu wird das Netz an dieser Stelle aufgespaltet indem zwei vollverkettete Ebenen parallel verwendet werden. Das Netz besitz somit zwei Ausgangsebenen. Diese Ebenen besitzen jeweils zwei Hidden Units um so eine zweidimensionale Gaussverteilung zu erzeugen.\\  
\\
Den Encoder soll Samples, gezogen aus dem Latentraum $z$ wieder auf die urspr�ngliche Eingabe zur�ck mappen. Dazu besitzt der Encoder erste Ebene eine Sampling Funktion, welche aus dem Input $\mu(X)$ und $\sigma^2(X)$ den Latentraum $z=\mu(X)+\Sigma^{\frac{1}{2}}(X)\epsilon$ erzeugt, sodass dieser differenzierbar bleibt. Die Implementierung wird in Algorithmus \ref{Sampling} gezeigt. Dabei wurde die Varianz als log-Varianz modelliert um ein numerisch stabileres Verhalten zum erzielen.\\
%
Nach der Sampling Ebene besitzt der Decoder zwei vollverkettete Ebenen mit 512 und 1152 Hidden Units. Um daraus wieder ein zweidimensionales Bild zu erhalten wird eine Reshape Ebene eingesetzt welche eine $3\times3\tims128$ Feature Map erzeugt. Anschlie�end werden drei Convolution Transposed Ebenen verwednet mit Kernel Gr��e $3\times3$ und Striding $2\times2$. Die Conv Transposed mit Striding sorgt daf�r, dass der Input wider geupsamplet wird um so die Ursprungsgr��e der Daten wieder zu erlangen. Zus�tzlich wird hier eine BatchNorm verwendet.


%
%
%
\begin{table}
\begin{tabular}{c c c c c c c}
\toprule
Operation&Kernel&Strides&Feature maps&BN?&Dropout&Nonlinarity\\
\midrule
Input&N/A&N/A&N/A&N/A&N/A&N/A\\
Convolution&3x3&2x2&32&\ding{55}&0.3&ReLU\\
Convolution&3x3&2x2&64&\ding{55}&0.3&ReLU\\
Convolution&3x3&2x2&128&\ding{55}&0.3&ReLU\\
Convolution&3x3&1x1&128&\ding{55}&0.3&ReLU\\
Flatten&N/A&N/A&N/A&\ding{55}&\ding{55}&N/A\\
Dense&N/A&N/A&512&\ding{55}&\ding{55}&ReLU\\
Dense $\mu(X)$&N/A&N/A&2&\ding{55}&\ding{55}&Linear\\
Dense $\sigma^2(X)$&N/A&N/A&2&\ding{55}&\ding{55}&Linear\\
\toprule
\end{tabular}
\caption{Encoder Architektur des VAE f�r den MNIST Datensatz}
\label{VAEencArchitecturMNIST}
\end{table}
%
%
\begin{table}
\begin{tabular}{c c c c c c c}
\toprule
Operation&Kernel&Strides&Feature maps&BN?&Dropout&Nonlinarity\\
\midrule
Input $\mu(X)$&N/A&N/A&N/A&N/A&N/A&N/A\\
Input $\sigma^2(X)$&N/A&N/A&N/A&N/A&N/A&N/A\\
Sampling&N/A&N/A&N/A&\ding{55}&\ding{55}&N/A\\
Dense&N/A&N/A&512&\ding{55}&\ding{55}&ReLU\\
Dense&N/A&N/A&1152&\ding{55}&\ding{55}&ReLU\\
Reshape&N/A&N/A&N/A&N/A&N/A&N/A\\
ConvTransposed&3x3&2x2&128&\ding{51}&0.3&Leaky ReLU\\
ConvTransposed&3x3&2x2&64&\ding{51}&0.3&Leaky ReLU\\
ConvTransposed&3x3&2x2&32&\ding{51}&0.3&Leaky ReLU\\
ConvTransposed&3x3&1x1&1&\ding{55}&0.3&Sigmoid\\
\toprule
\end{tabular}
\caption{Decoder Architektur des VAE f�r den MNIST Datensatz}
\label{VAEdecArchitecturMNIST}
\end{table}
%
%
\begin{algorithm}
\caption{Sampling}
\begin{algorithmic}
\REQUIRE $\mu(X),~\sigma^2(X)$
%\ENSURE $abc$
\STATE $\epsilon \sim \mathcal{N}(0,I)$
\STATE $z = \mu(X)+e^{\frac{\mathrm{log}(\sigma)}{2}}(X)\epsilon$
\end{algorithmic}
\label{Sampling}
\end{algorithm}
%
%
\section{GAN}
TODO
\section{Attak-Defence}
TODO